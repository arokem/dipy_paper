\documentclass{bioinfo}
\usepackage{url}

\usepackage[british,english]{babel}
\usepackage{mathpazo}
\usepackage[T1]{fontenc}
% \usepackage[latin9]{inputenc}
\usepackage{float}
\usepackage{amsmath}
\usepackage{graphicx}
\usepackage{setspace}
\usepackage{amssymb}
\usepackage{natbib}
\usepackage[title]{appendix}
\usepackage{siunitx}
\usepackage{chngcntr}
\usepackage{algorithmic}
\renewcommand{\algorithmicrequire}{\textbf{Input:}}
\renewcommand{\algorithmicensure}{\textbf{Output:}}

\usepackage{multirow}
\usepackage{rotating}

\makeatletter
\newfloat{algorithm}{H}{loa}[section]
\floatname{algorithm}{Algorithm}
\counterwithout{algorithm}{algorithm}
\def\argmin{\mathop{\operator@font arg\,min}} 
\def\argmax{\mathop{\operator@font arg\,max}} 
\makeatother

\copyrightyear{}
\pubyear{}

\begin{document}
\firstpage{1}

\title[QuickBundles]{Dipy, a library for the analysis of diffusion MRI data}

\author[Garyfallidis, Brett, Amirbekian, Rokem,  van der Walt, Descoteaux, 
Nimmo-Smith]{Eleftherios~Garyfallidis\,$^{1,2,*}$, Matthew~Brett\,$^{3}$,
  Bago~Amirbekian\,$^{4}$, Ariel~Rokem\,$^{5}$, Stefan~van der Walt\,$^{7}$, Maxime~Descoteaux\,$^{2}$, Ian~Nimmo-Smith\,$^{6}$ \footnote{to whom correspondence should be addressed. e-mail: garyfallidis@gmail.com}}

\address{\,$^{1}$University of Cambridge, Cambridge, UK\\
  \,$^{2}$University of Sherbrooke, Sherbrooke, CA\\ 
  \,$^{3}$University of California, Henry H. Wheeler, Jr. Brain Imaging Center, Berkeley, CA.\\
  \,$^{4}$University of California, San Francisco, CA, USA\\    
  \,$^{5}$Stanford University, Stanford, CA, USA\\
  \,$^{6}$MRC Cognition and Brain Sciences Unit, Cambridge, UK\\
  \,$^{7}$Stellenbosch University, Stellenbosch, South Africa\\  
  }


\history{}

\editor{}

\maketitle

\begin{abstract}
\noindent

Diffusion Imaging in Python (Dipy) is a free and open source software
project for the analysis of data from diffusion magnetic resonance
imaging (dMRI) experiments. DMRI is an application of MRI that can be
used to measure the micro-structure of the white matter in the human
brain in vivo. It utilizes the application of directionally oriented
magnetic gradients to estimate diffusion in different directions and
locations in the brain non-invasively. Many methods have been developed
to model the local configuration of nerve fibers in the white matter
based on this information and to infer the trajectory of fascicles
connecting different parts of the brain. However, no single open source
software platform gathers implementations of all these different
methods, where they can be easily understood and compared. Dipy aims to
provide transparent implementations for all the different steps of dMRI
analysis with a relatively uniform API.

Dipy implements classical signal reconstruction techniques, such as the
diffusion tensor model (Basser et al., 1994) and deterministic fiber
tractography (Mori et al., 1999). In addition, it implements cutting
edge novel reconstruction techniques, such as generalized Q imaging (Yeh
et al., 2010) and diffusion spectrum imaging with deconvolution
(Canales-Rodriguez et al. 2010), as well as methods for probabilistic
tracking (Berman et al., 2008) and unique methods for tractography
clustering (Garyfallidis et al., 2012). Many additional utility
functions, to calculate various statistics of dMRI data, visualization
functions, as well as file-handling routines exist to assist in the
development of novel techniques.
 
Dipy makes use of the scientific software for neuroimaging that exists
in python (e.g. nibabel), as well as python tools for numerical
processing (numpy), visualization (python-vtk), high performance
computation (cython) and software testing (nose).

In contrast to many other scientific software projects, dipy is not
being developed by a single research group. Rather, it is an open
project that encourages contributions from any scientist-developer
through the github Pull Request mechanism and open discussions on github
and on the project mailing list. Consequently, dipy has today an
international team of contributors, spanning 6 different academic
institutions in 4 countries and 3 continents, and still growing
(http://dipy.org).

\section{Keywords:} Python, Diffusion MRI, Diffusion Tensor model,
Deconvolution, Medical imaging, Open source software, Deterministic
tractography, Probabilistic tractography, Visualisation.

\end{abstract}

\section{Skeleton}

Here is a possible outline for the paper:

\begin{verbatim}
Philosophy/Mission
General Design Aspects
File Formats 
Preprocessing
   Eddy currents?
   Denoising?
Reconstruction
   DTI 
   DSI
   QBall
   Spherical Deconvolution 
Tracking
   Deterministic
   Probabilistic
Post-tracking
   Segmentation
   Track_counts
   Track Lengths & other statistics
   Connectivity matrix?
Conclusions
\end{verbatim}

\section{Philosophy and Mission}

Someone will write this.

This is to show how graphics (EPS) files are included. We use EPS for speed.

\begin{figure*}
\centerline{\includegraphics[width=160mm]{Figures/Fig_4_cst_simplification_relabeled_triple.eps}}
\caption{This is the figure caption - and a label to refer to it in the text \label{Fig:one}}
\end{figure*}

When we want to refer to this figure we use the label (see
Fig.~\ref{Fig:one}).

Here are some displayed equations (see Eq.~\ref{eq:direct_flip_distance}):
\begin{eqnarray}
  d_{\textrm{direct}}(s, t) = d(s, t) & = & \frac{1}{K}\sum_{i=1}^{K}|s_{i}-t_{i}|,\nonumber\\
  d_{\textrm{flipped}}(s, t) & = & d(s,t^F) = d(s^F,t),\nonumber\\
  \textrm{MDF}(s, t) & = & \min(d_{\textrm{direct}}(s, t), d_{\textrm{flipped}}(s, t))\label{eq:direct_flip_distance}.
\end{eqnarray}

Inline mathematics goes like this: $\frac{1}{K}\sum_{i=1}^{K}|s_{i}-t_{i}|$

Here we have an example of a table (see Table~\ref{Table_1}).

\begin{table}[th] \processtable{QB centroids performance compared with
random subsets\label{Table_1}} {\begin{tabular}{rrrr} %\hline Thresholds &
Comparison & Coverage \% (s.d.) & Overlap (s.d.) \\ \hline
\multirow{2}{*}{$10$~mm/$10$~mm} & QB Centroids & 99.96 (0.007) & 2.44
(0.08)\\ & Random & 90.49 (0.41) & 6.16 (0.55)\\ \hline
\multirow{2}{*}{$20$~mm/$20$~mm} & QB Centroids & 99.99 (0.004) & 3.54
(0.18)\\ & Random & 95.86 (0.62) & 6.81 (0.93)\\ \hline
\end{tabular}}{}
\end{table}

\section*{Acknowledgments}
Who do we need to acknowledge?

\section*{Disclosure/Conflict-of-Interest Statement}
Is it true that there are no conflicts of interest relating to this
work?

\selectlanguage{british}%
\bibliographystyle{apalike2}
%\bibliographystyle{plainnat}
%\bibliographystyle{IEEEabrv, IEEEtran}
%\bibliographystyle{IEEEtran}
%\bibliographystyle{elsarticle-harv}
\selectlanguage{english}
\bibliography{diffusion}

\end{document}
